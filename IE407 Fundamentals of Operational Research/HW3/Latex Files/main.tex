\documentclass[12pt]{article}
\usepackage{amsmath}
\usepackage{amsfonts}
\usepackage{graphicx}
\usepackage{hyperref}
\usepackage{geometry}
\usepackage{float}
\usepackage{listings}
\usepackage{color}
\usepackage{array}
\usepackage{booktabs}

\definecolor{mygray}{rgb}{0.5,0.5,0.5}
\definecolor{mymauve}{rgb}{0.58,0,0.82}

\lstset{
    language=Python,
    basicstyle=\ttfamily\footnotesize,
    numbers=left,
    numberstyle=\tiny\color{mygray},
    stepnumber=1,
    numbersep=5pt,
    showspaces=false,
    showstringspaces=false,
    showtabs=false,
    frame=single,
    tabsize=2,
    captionpos=b,
    breaklines=true,
    breakatwhitespace=false,
    title=\lstname,
    keywordstyle=\color{blue},
    commentstyle=\color{mygray},
    stringstyle=\color{mymauve},
}
\geometry{a4paper, margin=1in}

\title{\includegraphics{9.4.png}\\ IE407 - Homework 3 Report}
\author{Burak Yıldız - 2449049 \\ Ülkü Aktürk - 2450062}
\date{Due Date: 14.06.2024}
\begin{document}
\maketitle
\newpage

\section*{Question 1: Linear Programming Model for Tasch Co.}

\subsection*{Part (a): Linear Programming Model}

\textbf{Decision Variables:}
\begin{itemize}
    \item \( x_1 \): Number of PCs produced
    \item \( x_2 \): Number of Tablets produced
    \item \( x_3 \): Number of Microprocessors produced
\end{itemize}

\textbf{Constraints:}
\begin{align*}
    1. & \quad 5x_1 + 2x_2 + x_3 \leq 120 \quad \text{(Production Time Constraint)} \\
    2. & \quad 8x_1 + 4x_2 \leq 80 + x_3 \quad \text{(Microprocessor Usage Constraint)} \\
    3. & \quad x_1 \geq 0, \quad x_2 \geq 0, \quad x_3 \geq 0 \quad \text{(Non-negativity Constraints)}
\end{align*}

\textbf{Objective Function:}
\[
\text{Maximize} \quad Z = 50x_1 + 30x_2 - 2x_3
\]

\section*{Part (b): Simplex Method Solution}

\subsubsection*{Initial Simplex Tableau}

First, we convert the linear programming problem into standard form by introducing slack variables \(s_4\) and \(s_5\). The standard form is:

\[
\begin{aligned}
\text{Maximize } & F(x) = 50x_1 + 30x_2 - 2x_3 \\
\text{Subject to:} \\
& 5x_1 + 2x_2 + x_3 + s_4 = 120 \\
& 8x_1 + 4x_2 - x_3 + s_5 = 80 \\
& x_1, x_2, x_3, s_4, s_5 \geq 0
\end{aligned}
\]

We set up the initial simplex tableau:

\[
\begin{array}{c|cccccc|c}
 & x_1 & x_2 & x_3 & s_4 & s_5 & Z & \text{RHS} \\
\hline
s_4 & 5 & 2 & 1 & 1 & 0 & 0 & 120 \\
s_5 & 8 & 4 & -1 & 0 & 1 & 0 & 80 \\
\hline
Z & -50 & -30 & 2 & 0 & 0 & 1 & 0 \\
\end{array}
\]

\subsubsection*{Iteration 1}

\textbf{Step 1: Determine the Entering Variable}

 Here, \( x_1 \) has the most negative coefficient (-50), so \( x_1 \) is the entering variable.

\textbf{Step 2: Determine the Leaving Variable}

The smallest ratio is 10, so \( s_5 \) is the leaving variable.

\textbf{Pivot Operation}

\[
\begin{array}{c|cccccc|c}
 & x_1 & x_2 & x_3 & s_4 & s_5 & Z & \text{RHS} \\
\hline
s_4 & 5 & 2 & 1 & 1 & 0 & 0 & 120 \\
x_1 & 1 & \frac{1}{2} & -\frac{1}{8} & 0 & \frac{1}{8} & 0 & 10 \\
\hline
Z & -50 & -30 & 2 & 0 & 0 & 1 & 0 \\
\end{array}
\]

Next, we perform row operations

\[
\begin{array}{c|cccccc|c}
 & x_1 & x_2 & x_3 & s_4 & s_5 & Z & \text{RHS} \\
\hline
s_4 & 0 & -0.5 & 1.625 & 1 & -0.625 & 0 & 70 \\
x_1 & 1 & \frac{1}{2} & -\frac{1}{8} & 0 & \frac{1}{8} & 0 & 10 \\
\hline
Z & 0 & -5 & -3.25 & 0 & 6.25 & 1 & 500 \\
\end{array}
\]

\subsubsection*{Iteration 2}

\textbf{Step 1: Determine the Entering Variable}

 \( x_2 \) (-5), so \( x_2 \) is the entering variable.

\textbf{Step 2: Determine the Leaving Variable}

\( x_1 \) is the leaving variable.

\textbf{Pivot Operation}


\[
\begin{array}{c|cccccc|c}
 & x_1 & x_2 & x_3 & s_4 & s_5 & Z & \text{RHS} \\
\hline
s_4 & 0 & -0.5 & 1.625 & 1 & -0.625 & 0 & 70 \\
x_2 & 2 & 1 & -\frac{1}{4} & 0 & \frac{1}{4} & 0 & 20 \\
\hline
Z & 0 & -5 & -3.25 & 0 & 6.25 & 1 & 500 \\
\end{array}
\]

Next, we perform row operations

\[
\begin{array}{c|cccccc|c}
 & x_1 & x_2 & x_3 & s_4 & s_5 & Z & \text{RHS} \\
\hline
s_4 & 1 & 0 & 1.5 & 1 & -0.5 & 0 & 80 \\
x_2 & 2 & 1 & -\frac{1}{4} & 0 & \frac{1}{4} & 0 & 20 \\
\hline
Z & 10 & 0 & -4 & 0 & 7.5 & 1 & 600 \\
\end{array}
\]

\subsubsection*{Iteration 3}

\textbf{Step 1: Determine the Entering Variable}

The variable with the most negative coefficient in the Z-row is \( x_3 \) (-4), so \( x_3 \) is the entering variable.

\textbf{Step 2: Determine the Leaving Variable}

The smallest positive ratio is 53.33, so \( s_4 \) is the leaving variable.

\textbf{Pivot Operation}


\[
\begin{array}{c|cccccc|c}
 & x_1 & x_2 & x_3 & s_4 & s_5 & Z & \text{RHS} \\
\hline
s_4 & \frac{2}{3} & 0 & 1 & \frac{2}{3} & -\frac{1}{3} & 0 & \frac{80}{3} \\
x_2 & 2 & 1 & -\frac{1}{4} & 0 & \frac{1}{4} & 0 & 20 \\
\hline
Z & 10 & 0 & -4 & 0 & 7.5 & 1 & 600 \\
\end{array}
\]

Next, we perform row operations

\[
\begin{array}{c|cccccc|c}
 & x_1 & x_2 & x_3 & s_4 & s_5 & Z & \text{RHS} \\
\hline
s_4 & \frac{2}{3} & 0 & 1 & \frac{2}{3} & -\frac{1}{3} & 0 & \frac{80}{3} \\
x_2 & \frac{8}{3} & 1 & 0 & \frac{2}{3} & \frac{1}{12} & 0 & \frac{260}{3} \\
\hline
Z & \frac{38}{3} & 0 & 0 & \frac{8}{3} & 6 & 1 & \frac{893.33}{3} \\
\end{array}
\]

\subsubsection*{Optimal Solution}

The optimal solution is:

\[
\begin{aligned}
x_1 &= 0 \\
x_2 &= 33.33 \\
x_3 &= 53.33 \\
\end{aligned}
\]

The maximum profit is \( Z = 893.33 \).

\subsection*{Part (c): Solution Using Pyomo}
The code is appended at the end of the report.

\subsubsection*{Output of the code}

The output of the code is:

\begin{itemize}
    \item \(x_1\): 0.0
    \item \(x_2\): 33.3333333333333
    \item \(x_3\): 53.3333333333333
    \item Objective: 893.3333333333323
\end{itemize}


\section*{Question 2: Mixed-Integer Programming Model}

\section*{Branch and Bound Method}

\subsection*{Initial LP Relaxation}

We solve the initial LP relaxation by ignoring the integer constraints on \(x_2\) and \(x_3\):

\[
\begin{aligned}
\text{Maximize } & F(x) = 8x_1 + 6x_2 + 2x_3 \\
\text{Subject to:} \\
& 6x_1 + 4x_2 + 2x_3 \leq 14 \\
& 4x_1 + 2x_2 + 4x_3 \leq 22 \\
& x_1, x_2, x_3 \geq 0
\end{aligned}
\]

The initial simplex tableau is:

\[
\begin{array}{c|ccccc|c}
 & x_1 & x_2 & x_3 & x_4 & x_5 & \text{RHS} \\
\hline
x_4 & 6 & 4 & 2 & 1 & 0 & 14 \\
x_5 & 4 & 2 & 4 & 0 & 1 & 22 \\
\hline
Z & -8 & -6 & -2 & 0 & 0 & 0 \\
\end{array}
\]

\subsubsection*{Iteration 1}

\[
\begin{array}{c|ccccc|c}
\text{B} & C_b & P & x_1 & x_2 & x_3 & x_4 & x_5 & Q \\
\hline
x_4 & 0 & 14 & 8 & 6 & 2 & 0 & 0 & 2.33 \\
x_5 & 0 & 22 & 4 & 2 & 4 & 0 & 1 & 5.5 \\
\hline
\text{max} & & & -8 & -6 & -2 & 0 & 0 & 0 \\
\end{array}
\]

\subsubsection*{Iteration 2}

\[
\begin{array}{c|ccccc|c}
\text{B} & C_b & P & x_1 & x_2 & x_3 & x_4 & x_5 & Q \\
\hline
x_1 & 8 & 2.33 & 1 & 0.67 & 0.33 & 0.17 & 0 & 3.5 \\
x_5 & 0 & 12.67 & 0 & -0.67 & 2.67 & -0.67 & 1 & -19 \\
\hline
\text{max} & & & 0 & -0.67 & 0.67 & 1.33 & 0 & 18.67 \\
\end{array}
\]

\subsubsection*{Iteration 3}

\[
\begin{array}{c|ccccc|c}
\text{B} & C_b & P & x_1 & x_2 & x_3 & x_4 & x_5 & Q \\
\hline
x_1 & 8 & 0 & 1 & 0 & 0.5 & 0.25 & 0 & 0 \\
x_2 & 6 & 3.5 & 1.5 & 1 & 0.5 & 0.25 & 0 & 0 \\
x_5 & 0 & 15 & 1 & 0 & 0.5 & -0.5 & 1 & 0 \\
\hline
\text{max} & & & 0 & 1 & 0 & 1.5 & 0 & 21 \\
\end{array}
\]

\subsection*{Optimal Solution for LP Relaxation}

The optimal solution is:

\begin{itemize}
    \item \(x_1 = 0\)
    \item \(x_2 = 3.5\)
    \item \(x_3 = 0\)
    \item Objective: 21
\end{itemize}

\subsection*{Branch and Bound Process}

Since \(x_2 = 3.5\) is not an integer, we create two branches:
1. \(x_2 \leq 3\)
2. \(x_2 \geq 4\)

\subsubsection*{Branch 1: \(x_2 = 3\)}

For this branch, we add the constraint \( x_2 = 3 \) to the original problem:

\[
\begin{aligned}
\text{Maximize } & F(x) = 8x_1 + 6x_2 + 2x_3 \\
\text{Subject to:} \\
& 6x_1 + 2x_2 + x_3 \leq 14 \\
& 4x_1 + 2x_2 + 4x_3 \leq 22 \\
& x_2 = 3 \\
& x_1 \geq 0 \quad (x_1 \text{ continuous}) \\
& x_3 \geq 0 \quad (x_3 \text{ integer})
\end{aligned}
\]

The revised simplex tableau for this branch is:

\[
\begin{array}{c|ccccc|c}
 & x_1 & x_2 & x_3 & x_4 & \text{RHS} \\
\hline
x_3 & 0 & 2 & 6 & 1 & 0.33 \\
x_4 & 0 & 16 & 4 & 1 & 4 \\
\hline
Z & 0 & -8 & -2 & 0 & 0 \\
\end{array}
\]

\subsubsection*{Iteration 1}

\[
\begin{array}{c|ccccc|c}
\text{B} & C_b & P & x_1 & x_2 & x_3 & x_4 & Q \\
\hline
x_3 & 0 & 2 & 8 & 2 & 0 & 0.33 \\
x_4 & 0 & 16 & 4 & 1 & 0 & 4 \\
\hline
\text{max} & & & -8 & -6 & 0 & 0 \\
\end{array}
\]

\subsubsection*{Iteration 2}

\[
\begin{array}{c|ccccc|c}
\text{B} & C_b & P & x_1 & x_2 & x_3 & x_4 & Q \\
\hline
x_1 & 8 & 0.33 & 1 & 0.33 & 0.17 & 0 & 3.5 \\
x_4 & 0 & 14.67 & 0 & 2.67 & -0.67 & 1 & -19 \\
\hline
\text{max} & & & 0 & 0.67 & 1.33 & 0 & 18.67 \\
\end{array}
\]

\subsubsection*{Iteration 3}

\[
\begin{array}{c|ccccc|c}
\text{B} & C_b & P & x_1 & x_2 & x_3 & x_4 & Q \\
\hline
x_1 & 8 & 0 & 1 & 0.5 & 0.25 & 0 & 0 \\
x_2 & 6 & 3.5 & 1.5 & 0.5 & 0.25 & 0 & 0 \\
x_5 & 0 & 15 & 1 & 0 & -0.5 & 1 & 0 \\
\hline
\text{max} & & & 0 & 1 & 0 & 0 & 21 \\
\end{array}
\]

The solution for this branch is:

\begin{itemize}
    \item \(x_1 = 0.33\)
    \item \(x_2 = 3\)
    \item \(x_3 = 0\)
    \item Objective: 20.67
\end{itemize}

\subsubsection*{Branch 2: \(x_2 = 4\)}

For this branch, we add the constraint \( x_2 = 4 \) to the original problem: \\

Even if the other variables are zero is $x_2$ is 4 the second constraint is not satisfied so we eliminate this branch.
\subsection*{Comparison of Solutions}

We compare the objective values from both branches:

\[
\begin{aligned}
\text{Initial branch:} & \quad Z = 21 \quad \text{for} \quad x_1 = 0, x_2 = 3.5, x_3 = 0 \\
\text{Branch 1:} & \quad Z = 20.67 \quad \text{for} \quad x_1 = 0.33, x_2 = 3, x_3 = 0 \\
\text{Branch 2:} & \quad Z = 4 \quad \text{for} \quad x_1 = 0, x_2 = 4, x_3 = 0 \\

\end{aligned}
\]

Branch 2 does not satisfy the constraints so that branch is eliminated and in the initial branch $x_2$ is not an integer so we choose branch 1 for optimal solution.

\subsection*{Optimal Solution}

The optimal solution is:

\begin{itemize}
    \item \(x_1 = 0.33\)
    \item \(x_2 = 3\)
    \item \(x_3 = 0\)
    \item Objective: 20.67
\end{itemize}

\subsection*{Part (b): Solution Using Pyomo}

\subsubsection*{Output of the code}

The code is appended at the end of the report ,the output of the code is:

\begin{itemize}
    \item \(x_1\): 0.333333333333333
    \item \(x_2\): 3.0
    \item \(x_3\): 0.0
    \item Objective: 20.666666666666664
\end{itemize}

\newpage
\appendix
\section*{Appendix}

\subsection{Question 1: Initial Model Code}

\lstinputlisting{tasch_co_lp.py}

\newpage

\subsection{Question 2: Initial Model Code}
\lstinputlisting{mip_model.py}

\end{document}

